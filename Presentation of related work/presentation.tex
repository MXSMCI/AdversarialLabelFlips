% !TEX encoding = UTF-8 Unicode
\documentclass[10pt,usepdftitle=false,aspectratio=169]{beamer}
\usepackage[left]{muibkitsec}
\usepackage{listings}

\usepackage{microtype}
\usepackage{graphbox}
\usepackage{booktabs} 

\usepackage{amsmath,amssymb,amsfonts,amsthm,mathtools}
\usepackage{algorithmic}
\usepackage{textcomp}
\usepackage{xcolor}
\usepackage{diagbox}
\usepackage{float, multirow}
\usepackage{tikz, pgfplots}
\usepackage{tikzsymbols}
\usetikzlibrary{spy}
\usepackage{subcaption}
\usepgfplotslibrary{groupplots}
\pgfplotsset{compat=newest}

% ------------------------------------------------------------------------
\title{Adversarial Label Flips}
%\subtitle{modified new beamer template}
\author{Matthias Dellago \& Maximilian Samsinger}
\date{22 January 2021}

% ------------------------------------------------------------------------

\begin{document}
\DeclarePairedDelimiter\abs{\lvert}{\rvert}%
\DeclarePairedDelimiter\norm{\lVert}{\rVert}%
\DeclarePairedDelimiter\ceil{\lceil}{\rceil}
\DeclarePairedDelimiter\floor{\lfloor}{\rfloor}

\begin{frame}[plain]
	\maketitle
\end{frame}	

%1x Introduction
%1x Slide Angriffe
%1x Slide Pytorch
%1x Slide Foolbox 
%1x Slide Datensätze
%1x Slide Pytorch
%
%
%1x Slide Architectures

\begin{frame}[fragile]
	\frametitle{Introduction/Reminder from last time}
	Panda slide from last time
\end{frame}

\begin{frame}[fragile]
	\frametitle{Introduction/Reminder from last time}
	\begin{columns}
		\begin{column}{.5\columnwidth}
			\begin{block}{What do we want to do?}
				Generate adversarial examples. Create confusion matrix
			\end{block}
		\end{column}
		\begin{column}{.5\columnwidth}
			\begin{exampleblock}{Confusion matrix}
				Confusion matrix from last presentation
			\end{exampleblock}
		\end{column}
	\end{columns}
\end{frame}

\begin{frame}[fragile]
	\frametitle{Adversarial examples + some attacks}
	\begin{columns}
		\begin{column}{.5\columnwidth}
			\begin{block}{Block}
				Adversarial examples have been introduced in \cite{Szegedy13}.
			\end{block}
		\end{column}
		\begin{column}{.5\columnwidth}
			\begin{alertblock}{Examples of attacks: FGSM + PGD}
				FGSM has been introduced in \cite{goodfellow2014explaining}. (Panda example) \\
				An iterated version (PGD) was introduced in \cite{madry2017towards}. 
			\end{alertblock}
		\end{column}
	\end{columns}
\end{frame}

\begin{frame}[fragile]
	\frametitle{How does FGSM work?}
		\begin{block}{FGSM}
			text
		\end{block}
	\pause For more attacks we use Foolbox.
\end{frame}


\begin{frame}[fragile]
	\frametitle{What is Foolbox?}
	\begin{columns}
		\begin{column}{.5\columnwidth}
			\begin{block}{Foolbox}
				A suit of attacks is available with FoolBox! \cite{rauber2017foolbox}. Maybe show the full list of attacks directly on the website.
				\url{https://foolbox.readthedocs.io/en/stable/modules/attacks.html}
			\end{block}
		\end{column}
		\begin{column}{.5\columnwidth}
			\begin{alertblock}{Implementation}
				Foolbox can be used with PyTorch, TensorFlow and JAX. We arbitrarily choose Pytorch!
			\end{alertblock}
		\end{column}
	\end{columns}
\end{frame}

\begin{frame}[fragile]
	\frametitle{Optional Slide 1: Data set}
	Some images for MNIST, Fashion-MNIST and CIFAR-10.
\end{frame}

\begin{frame}[fragile]
	\frametitle{Optional Slide 2: Convolutional neural networks}
	We use small convolutional neural networks \cite{lecun1999object} for the "easy" data sets. For CIFAR-10 we will use ResNet-18, a residual neural network \cite{he2016deep}, \cite{he2016identity}.
\end{frame}

\begin{frame}[allowframebreaks]
	\frametitle{References}
	\bibliographystyle{unsrt}
	\bibliography{literature}
\end{frame}


\end{document}
